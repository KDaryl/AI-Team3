\hypertarget{_hud_8cpp}{}\section{Source/\+Hud.cpp File Reference}
\label{_hud_8cpp}\index{Source/Hud.cpp@{Source/Hud.cpp}}
{\ttfamily \#include \char`\"{}Hud.\+h\char`\"{}}\newline
